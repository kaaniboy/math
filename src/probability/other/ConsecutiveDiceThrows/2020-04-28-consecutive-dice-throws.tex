\documentclass{article}
\usepackage{amsmath}
\usepackage{graphicx}
\usepackage[utf8]{inputenc}
\usepackage{parskip}

\setcounter{secnumdepth}{0}

\date{}
\author{Kaan Aksoy | April 28, 2020}
\title{}

\begin{document}

\maketitle
\section{Consecutive Dice Throws}
\subsection{Problem}

What is the expected number of times one must roll a fair dice before 
getting $2$ consecutive $6$'s?

\subsection{Solution}
According to the geometric distribution, we can expect $6$ rolls before 
landing the first $6$. At that point, there is a $\frac{1}{6}$ chance 
that the next roll is a $6$, which would result in two consecutive $6$'s. 
On the other hand, there is a $\frac{5}{6}$ chance of landing a number that is 
not $6$, at which point we must reset our count. This formulation can be 
represented by the following recurrence, where $X$ is a random variable 
representing the number of rolls for $2$ consecutive $6$'s:

$$E[X] = 6 + \frac{1}{6}(1) + \frac{5}{6}(1+E[X])$$
$$E[X] = 42$$

Thus, the expected number of rolls to land $2$ consecutive $6$'s is $42$.

This approach can be generalized to an arbitrary number of consecutive $6$'s. 
For example, suppose $Y$ represents the number of rolls for $3$ consecutive $6$'s. 
Knowing that $E[X] = 42$, we can write a recurrence for $E[Y]$ as follows:

$$E[Y] = E[X] + \frac{1}{6}(1) + \frac{5}{6}(1+E[Y])$$
$$E[Y] = 258$$

\end{document}
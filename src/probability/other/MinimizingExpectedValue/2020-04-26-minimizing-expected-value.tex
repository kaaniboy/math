\documentclass{article}
\usepackage{amsmath}
\usepackage{graphicx}
\usepackage[utf8]{inputenc}
\usepackage{parskip}

\setcounter{secnumdepth}{0}

\date{}
\author{Kaan Aksoy | Feb 20, 2020}
\title{}

\begin{document}

\maketitle
\section{Minimizing Expected Value}
\subsection{Problem}

Suppose $X$ is a random variable with $E[X^2] < \infty$. What is 
the constant $c$ that minimizes $E[(X - c)^2]$?

\subsection{Solution}

As with most optimization problems, we will begin by taking the 
derivative of $E[(X - c)^2]$ with respect to $c$. To do so, we replace 
the expectation with its definition, letting $f(x)$ represent the 
probability density function of $X$.

\begin{equation*}
\begin{split}
\dfrac{d}{dc}E\left[(X - c)^2\right] &= 
\dfrac{d}{dc} \int_{-\infty}^{\infty} (x-c)^2f(x)dx \\
&= -2\int_{-\infty}^{\infty}(x-c)f(x)dx \\
&= -2\left[ \int_{-\infty}^{\infty}xf(x)dx - 
\int_{-\infty}^{\infty}cf(x)dx \right] \\
&= -2\left( E[X] - c \right) \\
\end{split}
\end{equation*}

Next, we set the derivative equal to $0$ and solve for $c$:

$$ -2\left( E[X] - c \right) = 0$$
$$ c = E[X]$$

Thus, $E[(X - c)^2]$ is minimized by $c = E[X]$. We confirm this by noting 
$\dfrac{d^2}{dc}E[(X - c)^2] = 2$, indicating upward concavity.

\subsection{Alternative Approach}
There is another, cleaner approach to finding the solution. We begin 
by utilizing the linearity of expectation property:

$$E[(X - c)^2] = E[X^2 - 2cX + c^2] = E[X^2] - 2cE[X] +c^2$$

Then, we take the derivative of this new formulation and set the result to $0$:

$$\dfrac{d}{dc} \left[ E[X^2] - 2cE[X] +c^2 \right] = 0$$
$$-2E[X] + 2c = 0$$
$$c = E[X]$$

\end{document}
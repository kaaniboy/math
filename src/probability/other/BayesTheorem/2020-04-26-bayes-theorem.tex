\documentclass{article}
\usepackage{amsmath}
\usepackage{graphicx}
\usepackage[utf8]{inputenc}
\usepackage{parskip}%

\setcounter{secnumdepth}{0}

\date{}
\author{Kaan Aksoy | April 26, 2020}
\title{}

\begin{document}

\maketitle

\section{Bayes' Theorem}

Bayes' theorem is central to probability. It serves as the 
foundation for \textit{statistical inference}, which is used to estimate 
population parameters based on observations. For example, Bayes' 
theorem may be applied to find the probability that a coin is fair, 
given that all prior flips have landed on heads.

Mathematically, Bayes' theorem may be stated as follows for 
events $A$ and $B$:

$$
P(A|B) = \frac{P(B|A)P(A)}{P(B)}
$$

In the above, $P(A)$ is called the \textit{prior}, and represents  
the probability of $A$ before evidence is incorporated. Likewise, 
$P(A|B)$ is called the \textit{posterior}, and represents the 
probability of $A$ after evidence is incorporated. 

\subsection{Proof}
It is easy to arrive at this theorem from the following 
two formulations of the definition of conditional probability:

$$P(A|B) = \frac{P(A \cap B)}{P(B)}$$
$$P(A \cap B) = P(A)P(B|A)$$
$$\Downarrow$$
$$P(A|B) = \frac{P(B|A)P(A)}{P(B)}$$

\end{document}
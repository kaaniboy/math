\documentclass{article}
\usepackage{amsmath}
\usepackage{graphicx}
\usepackage[utf8]{inputenc}
\usepackage{parskip}
\usepackage[symbol]{footmisc}

\renewcommand{\thefootnote}{\fnsymbol{footnote}}
\setcounter{secnumdepth}{0}

\date{}
\author{Kaan Aksoy | Feb 24, 2020}

\begin{document}

\maketitle
\section{Collecting Coupons}
\subsection{Problem}

Coupons in cereal boxes are numbered $1$ to $5$, 
and a set of one of each is required for a prize. 
With one coupon per box, how many boxes on the 
average are required to make a complete set?

\subsection{Solution}

Let $X_{1}$, $X_{2}$, \ldots, $X_{5}$ be random 
variables representing the number of boxes opened 
until receiving the $1$\textsuperscript{st},
$2$\textsuperscript{nd}, \ldots, 
and $5$\textsuperscript{th} unique coupon, respectively, 
where the number of boxes is counted since the last 
box containing a unique coupon was opened. For example, $X_{3}$ 
represents the number of boxes opened after the 
$2$\textsuperscript{nd} unique coupon was found up 
until the $3$\textsuperscript{3rd} unique coupon was found.

Each $X_{i}$ is a 
geometric random variable associated with a 
different probability of success. For example, $X_{1}$ 
is associated with $p_{1} = 1$ because the 
$1$\textsuperscript{st} box will obviously contain
a unique coupon. $X_{2}$ is 
associated with $p_{2} = \frac{4}{5}$ because only $4$
of the $5$ coupons are unique at that point. 
Probabilities for the remaining $X_{i}$ can be 
calculated in a similar fashion. For geometric 
random variables, we also know that 
$E[X_{i}] = \frac{1}{p_{i}}$.

To solve the problem, we can apply the 
\textit{Linearity of Expectation} property as 
follows:

\begin{equation*}
\begin{split}
E \left[\sum_{i=1}^5 X_{i} \right] &= \sum_{i=1}^5 E[X_{i}] \\
&= \sum_{i=1}^5 \frac{1}{p_{i}} \\
&= 5 \left( \frac{1}{5} + \frac{1}{4} + 
\frac{1}{3} + \frac{1}{2} + \frac{1}{1} \right) \\
&= \frac{137}{12} \approx 11.42
\end{split}
\end{equation*}

\subsection{Notes}

When generalized to $n$ coupons, the solution to this 
problem can be written as:

$$
C(n) = n \sum_{i=1}^n \frac{1}{i} = nH_{n}
$$

In the above, $H_{n}$ represents the $n$\textsuperscript{th} 
\textit{Harmonic Number}, which is the sum of the reciprocals 
of integers from $1$ to $n$.

\end{document}
\documentclass{article}
\usepackage{amsmath}
\usepackage{graphicx}
\usepackage[utf8]{inputenc}
\usepackage{parskip}
\usepackage[symbol]{footmisc}

\renewcommand{\thefootnote}{\fnsymbol{footnote}}
\setcounter{secnumdepth}{0}

\date{}
\author{Kaan Aksoy | March 9, 2020}

\begin{document}

\maketitle
\section{The Cliff-Hanger}
\subsection{Problem}

From where he stands, one step toward a cliff would send a drunken 
man over the edge.  He takes random steps, either toward or away 
from the cliff. At any step, his probability of taking a step away 
is $p=\frac{2}{3}$, and his probability of taking a step toward 
the cliff is $1-p=\frac{1}{3}$. What is his chance of escaping 
the cliff?

\subsection{Solution}

We begin by representing the cliff as a number line, where $x=0$ is 
the edge, $x=1$ is the man's starting point, $x=2$ is $2$ steps 
from the edge, and so on:

\begin{figure}[h]
    \centering
    \includegraphics[width=4cm]{Problem35_CliffHanger.png}
    \caption{Number line representation of cliff}
\end{figure}

Let $P_1$ be the probability that the man eventually reaches $x=0$ 
(the edge) starting at $x=1$. We can write $P_1$ as follows:

$$P_1 = (1-p) + pP_2$$

In the above, $(1-p)$ represents the probability of immediately taking 
a step towards the edge. The latter part represents the probability of 
taking a step away and then eventually reaching $x=0$. Here, as before, 
$P_2$ represents the probability that the man eventually reaches $x=0$ 
starting at $x=2$.

We can continue expanding this equation recursively. For example, $P_2$ 
can be written as $P_2 = (1-p)P_1 + pP_3$. However, this does not lead 
us closer to the solution.

An important insight to observe is that $P_2 = P_1P_1 = P_1^2$. The first 
$P_1$ is equivalent to the probability of moving from $x=2$ to $x=1$, and 
the second $P_1$ is used in the traditional sense, representing 
the probability of moving from $x=1$ to $x=0$. We can plug this 
identity into the original equation and apply the quadratic formula to get 
the following:

\begin{equation*}
\begin{split}
P_1 &= (1-p) + pP_1^2 \\ 
&= \frac{1 \pm \sqrt{1 - 4p(1-p)}}{2p} \\
&= \frac{1 \pm (1-2p)}{2p} \\
&= 1, \frac{1-p}{p}
\end{split}
\end{equation*}

Applying the solution $P_1 = \frac{1-p}{p}$ to $p=\frac{2}{3}$ 
gives a probability of $1-P_1 = \frac{1}{2}$ of the man escaping 
the cliff.

\end{document}
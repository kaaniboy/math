\documentclass{article}
\usepackage{amsmath}
\usepackage{graphicx}
\usepackage[utf8]{inputenc}
\usepackage{parskip}
\usepackage[symbol]{footmisc}

\newtheorem{theorem}{Theorem}[section]
\renewcommand{\thefootnote}{\fnsymbol{footnote}}

\setcounter{secnumdepth}{0}

\date{}
\author{Kaan Aksoy | March 10, 2020}

\begin{document}

\maketitle
\section{The First Ace}
\subsection{Problem}
Shuffle an ordinary deck of $52$ playing cards containing $4$ 
aces. Then turn up cards from the top until the first ace 
appears. On the average, how many cards are required to produce 
the first ace?

\subsection{Solution}

Let $X$ represent the number of cards that are turned 
up to produce the $1$\textsuperscript{st} ace. For this 
problem, we cannot apply the \textit{Geometric Distribution}
because cards are sampled without replacement.

Instead, we begin by considering the 
probabilities of drawing the $1$\textsuperscript{st} ace 
on the $1$\textsuperscript{st} card, $2$\textsuperscript{nd} 
card, and so on:

\begin{equation*}
\begin{split}
P(1st\;Card) &= \frac{4}{52} \\
P(2nd\;Card) &= \left(\frac{48}{52}\right)\left(\frac{4}{51}\right) \\
P(3rd\;Card) &= \left(\frac{48}{52}\right)\left(\frac{47}{51}\right)
\left(\frac{4}{50}\right) \\
P(n^{th}\; card) &= 4\left[\frac{48!}{(49-x)!}\right]\left[{\frac{(52-x)!}{52!}} \right]
\end{split}
\end{equation*}

Then, we can calculate the average number of cards by applying the 
definition of expected value:

$$E[X] = \sum_{x=1}^{52} 
4x\left[\frac{48!}{(49-x)!}\right]\left[{\frac{(52-x)!}{52!}} \right]
= \frac{53}{5} = 10.6$$

Thus, on average it will take $10.6$ cards to get the 
$1$\textsuperscript{st} ace.

\subsection{Alternative Solution}
The solution above is complex due to the unwieldy summation. Another 
approach is to apply the \textit{Principle of Symmetry}, which states 
that $n$ randomly placed points will divide a segment into $n+1$ pieces, 
each of which has the same distribution.

This problem is an application of the principle with $n=4$, since each 
ace in the deck represents a division point. Then, the average length 
of the $5$ segments (stretches of cards without an ace) is 
$\frac{52-4}{5} = \frac{48}{5}$. Each of these segments is immediately 
followed by an ace, so the expected number of cards until the 
$1$\textsuperscript{st} ace is the following:

$$E[X] = \frac{48}{5} + 1 = \frac{53}{5} = 10.6$$

\end{document}
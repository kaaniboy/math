\documentclass{article}
\usepackage[utf8]{inputenc}
\usepackage{parskip}%

\setcounter{secnumdepth}{0}

\date{}
\author{Kaan Aksoy | Feb 10, 2020}

\begin{document}

\maketitle

\section{The Flippant Juror}

The 3-person jury makes its decisions based on majority rule. Jurors $A$ 
and $B$ each have probability $p$ of making the right decision, while 
Juror $C$ has a probability of $\frac{1}{2}$ of making the right decision. 
The cases where they make the right decision are as follows (where 
$R$ represents a right decision, and $W$ represents a wrong one):

\begin{center}
\begin{tabular}{ |c|c|c| }
 \hline
 Juror A & Juror B & Juror C \\ 
 \hline
 R & R & R \\  
 R & R & W \\ 
 R & W & R \\  
 W & R & R \\  
 \hline
\end{tabular}
\end{center}

Summing the probabilities of these 4 individual cases gives:

$$ \frac{p^2}{2} + \frac{p^2}{2} + \frac{p(1-p)}{2} + \frac{p(1-p)}{2} = p $$

Thus, the 1-person and 3-person juries both have the same probability 
of making the right decision.

\end{document}
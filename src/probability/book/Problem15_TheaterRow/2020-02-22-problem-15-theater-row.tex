\documentclass{article}
\usepackage{amsmath}
\usepackage{graphicx}
\usepackage[utf8]{inputenc}
\usepackage{parskip}
\usepackage[symbol]{footmisc}

\renewcommand{\thefootnote}{\fnsymbol{footnote}}
\setcounter{secnumdepth}{0}

\date{}
\author{Kaan Aksoy | Feb 22, 2020}

\begin{document}

\maketitle
\section{Theater Row}
\subsection{Problem}
Eight eligible bachelors and seven beautiful models happen randomly to have purchased single seats in the same 15-seat row of a theater. On the average, how many pairs of adjacent seats are ticketed for marriageable couples\footnote[1]{According to outdated cultural norms from the time of this problem's publication.}?

\subsection{Solution}

Let $X$ be a random variable denoting the number of pairs that are marriageable. 
We can rewrite $X$ as $X = X_{1} + X_{2} + \ldots 
+ X_{14}$, where each $X_{i}$ is a random variable 
representing the number of marriageable couples in 
the $i^{th}$ pair of adjacent seats. Note that each 
$X_{i}$ can take on either $0$ or $1$ only.

We can easily calculate $E[X_{i}]$ by considering the 
two arrangements that result in a marriageable pair, which 
are \textit{Male-Female} and \textit{Female-Male}:

$$
E[X_{i}] = \left(\frac{8}{15}\right) 
\left(\frac{7}{14}\right) + 
\left(\frac{7}{15}\right)
\left(\frac{8}{14}\right) = \frac{8}{15}
$$

Then, applying the \textit{Linearity of Expectation} property, we can 
calculate $E[X]$ as follows:

\begin{equation*}
\begin{split}
E[X] &= E[X_{1} + X_{2} + \ldots + X_{14}] \\
&= E[X_{1}] + E[X_{2}] + \ldots + E[X_{14}] \\
&= 14\left(\frac{8}{15}\right)
\end{split}
\end{equation*}

Thus, the expected number of marriageable pairs in the row of $15$ 
seats is $E[X] \approx 7.46$.
\end{document}
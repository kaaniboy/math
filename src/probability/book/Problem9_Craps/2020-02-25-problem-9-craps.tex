\documentclass{article}
\usepackage{amsmath}
\usepackage{graphicx}
\usepackage[utf8]{inputenc}
\usepackage{parskip}
\usepackage[symbol]{footmisc}

\renewcommand{\thefootnote}{\fnsymbol{footnote}}
\setcounter{secnumdepth}{0}

\date{}
\author{Kaan Aksoy | Feb 25, 2020}

\begin{document}

\maketitle
\section{Craps}
\subsection{Problem}
The game of craps, played with two die, is one of 
America's fastest and most popular games. Calculating 
the odds associated with it is an instructive exercise.

The rules are these. Only totals for the two dice count. 
The player throws the dice and wins at once if the total 
for the first throw is $7$ or $11$, loses at once if it is 
$2$, $3$ or $12$. Any other throw is called his "point". If 
the first throw is a point, the player throws the dice 
repeatedly until he either wins by throwing his point 
again or loses by throwing $7$. What is the player's chance 
to win?			

\subsection{Solution}

Let $X$ be a random variable representing the sum of 
the values on the two die, which has the range 
$\{2,\ldots,12\}$. By counting the relevant outcomes and dividing
by the total number of outcomes in the sample space ($36$), we 
get the following probabilities for $X$:

\vspace{0.3cm}
\begin{center}
\begin{tabular}{ |c|c|c|c|c|c|c|c|c|c|c|c| } 
 $x$ & $2$ & $3$ & $4$ & $5$ & $6$ & $7$ & $8$ & $9$ & $10$ & $11$ & $12$ \\
 \hline
 $P(X=x)$ & $\frac{1}{36}$ & $\frac{1}{18}$ & $\frac{1}{12}$ & 
 $\frac{1}{9}$ & $\frac{5}{36}$ & $\frac{1}{6}$ & 
 $\frac{5}{36}$ & $\frac{1}{9}$ & $\frac{1}{12}$ & 
 $\frac{1}{18}$ & $\frac{1}{36}$ \\
\end{tabular}
\end{center} 
\vspace{0.3cm}

In the range of $X$, the values that do not result in an 
immediate win or loss are $C = \{4,5,6,8,9,10\}$. 
For these cases, we must calculate the probability $D_x$ of drawing the "point" $x$ 
instead of a $7$ (which can be made in $6$ ways). We can disregard all other 
values in the sample space of $X$ because they do not result in the 
game terminating:

$$D_4 = \frac{3}{3+6} = \frac{1}{3}$$
$$D_5 = \frac{4}{4+6} = \frac{2}{5}$$
$$D_6 = \frac{5}{5+6} = \frac{5}{11}$$
$$D_8 = \frac{3}{3+6} = \frac{5}{11}$$
$$D_9 = \frac{4}{4+6} = \frac{2}{5}$$
$$D_{10} = \frac{3}{3+6} = \frac{1}{3}$$

This approach of only considering certain outcomes in the sample 
space is called the \textit{Method of Reduced Sample Space}.

Using these probabilities, we can calculate the player's 
chance of winning as follows:

\begin{equation*}
\begin{split}
P(Win) &= P(X=7) + P(X=11) + \sum_{x \in C} P(X=x)D_x \\
&\approx 0.493
\end{split}
\end{equation*}

Thus, craps is not a fair game, although it is quite 
close to being so.

\subsection{Alternative Solution}
We can also calculate the probabilities $D_x$ from above 
using the geometric distribution. 

Let $P_x$ representing the probability of the die sum not being 
$7$ (which results in a loss) and not being the "point" $x$ (which results 
in a win). Then, the probability of winning on the $(i + 1)$\textsuperscript{th} 
roll can be written as $(P_x)^i*P(X=x)$. Then, we can calculate 
$D_x$ by summing this infinite geometric series for all non-negative 
values of $i$:

$$
D_x = \frac{a_1}{1-r} = \frac{P(X=x)}{1 - P_x}
$$

As an example, we can calculate $D_4$ as follows:

$$
D_4 = \frac{\frac{1}{12}}{1-\left( 1 - \frac{1}{12} - \frac{1}{6} \right)} 
= \frac{1}{3}
$$

\end{document}
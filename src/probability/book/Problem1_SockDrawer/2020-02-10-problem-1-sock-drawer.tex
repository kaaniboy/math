\documentclass{article}
\usepackage[utf8]{inputenc}
\usepackage{parskip}%

\setcounter{secnumdepth}{0}

\date{}
\author{Kaan Aksoy | Feb 10, 2020}
\title{The Sock Drawer}

\begin{document}

\maketitle
\subsection{Part A}

We would like to find the fewest number of socks (which are either red or black) such that
the probability of two drawn socks both being red is $\frac{1}{2}$. 
Let $r$ be the number of red socks and $b$ be the number of 
black socks. 

The probability that the two drawn socks are both red is:
$$\frac{r}{r+b}\cdot\frac{r-1}{r+b-1} = \frac{1}{2}$$

Note that, for $b > 0$, $\frac{r}{r+b} > \frac{r-1}{r+b-1}$.
Using this, we can say that:

$$\left( \frac{r}{r+b} \right)^2 > \frac{1}{2} 
> \left( \frac{r-1}{r+b-1} \right)^2$$

Taking the square root and rearranging gives:
    
$$\frac{1-\sqrt{2}-b}{1-\sqrt{2}} > r > \frac{b}{\sqrt{2}-1}$$

Trying $b=1$, we get $3.414 > r > 2.414$, which leaves only 
$r=3$. Thus, the fewest number of socks for a $\frac{1}{2}$ 
probability of drawing two reds is $4$ 
(with $3$ red socks, and $1$ black sock).


\subsection{Part B}
Now, we would like to solve the same question with the added 
constraint that the number of black socks is even. To do so,
we will try increasing even values for $b$ along with the 
inequality from \textit{Part A}.

\begin{center}
\begin{tabular}{ |c|c|c|c| }
 \hline
 $b$ & Inequality & Potential $r$ & Probability of two reds \\ 
 \hline
 2 & $ 4.828 < r < 5.828 $ & 5 & 0.476 \\  
 4 & $ 9.656 < r < 10.656 $ & 10 & 0.494 \\
 6 & $ 14.485 < r < 15.485 $ & 15 & 0.50 \\
 \hline
\end{tabular}
\end{center}

Thus, The fewest number of socks with an even number of black 
socks is 21 (with 15 reds and 6 blacks).

\end{document}
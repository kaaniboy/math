\documentclass{article}
\usepackage{amsmath}
\usepackage{graphicx}
\usepackage[utf8]{inputenc}
\usepackage{parskip}
\usepackage[symbol]{footmisc}

\renewcommand{\thefootnote}{\fnsymbol{footnote}}
\setcounter{secnumdepth}{0}

\date{}
\author{Kaan Aksoy | Feb 25, 2020}

\begin{document}

\maketitle
\section{The Prisoner's Dilemma}
\subsection{Problem}
Three prisoners, $A$, $B$ and $C$, with apparently equally good 
records have applied for parole. The parole board has decided to 
release two of the three, and the prisoners know this but not which 
two. A warder friend of prisoner $A$ knows who are to be released. 
Prisoner A realizes that it would be unethical to ask the warder if 
he, $A$, is to be released, but thinks of asking for the name of one 
prisoner other than himself who is to be released. He thinks that before 
he asks, his chances of release are $\frac{2}{3}$. He thinks that if the 
warder says "$B$ will be released", his own chances have now gone down 
to $1/2$, because either $A$ and $B$ or $B$ and $C$ are to be released. 
And so $A$ decides not to reduce his chances by asking.  However, $A$ 
is mistaken in his calculations. Explain.

\subsection{Solution}

Prisoner $A$'s mistake is not considering the entire sample space of pairs of 
prisoners that could be released, which is $\Omega = \{AB, AC, BC\}$. In the case of 
$AB$, the warden will say that $B$ will be released, and $A$ has a $100\%$ chance 
of release. Similarly, in the case of $AC$, the warden will say that $C$ will be 
released, and $A$ has a $100\%$ chance of release. In the case of $BC$, the warden 
will say that either $B$ or $C$ will be released, and $A$ has a $0\%$ chance of 
release. Each outcome has a $\frac{1}{3}$ chance of occurring. When combined, this 
gives the overall chance of release with the warden's statement as:

$$P(A\;Released) = \left(\frac{1}{3}\right)1 + \left(\frac{1}{3}\right)1
+ \left(\frac{1}{3}\right)0 = \frac{2}{3}$$

Thus, asking the warden has no effect on prisoner $A$'s chances of 
being released.
\end{document}
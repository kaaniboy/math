\documentclass{article}
\usepackage{amsmath}
\usepackage{graphicx}
\usepackage[utf8]{inputenc}
\usepackage{parskip}
\usepackage[symbol]{footmisc}

\renewcommand{\thefootnote}{\fnsymbol{footnote}}
\setcounter{secnumdepth}{0}

\date{}
\title{}
\author{Kaan Aksoy | Mar 27, 2020}

\begin{document}
\maketitle

\section{The Ballot Box}

\subsection{Problem}
In an election, $2$ candidates, Albert and Benjamin, have in a ballot 
box $a$ and $b$ votes respectively, with $a > b$. For example, $a=3$ 
and $b=2$. If ballots are randomly drawn and tallied, what is the 
chance that at least once after the first tally the candidates have 
the same number of tallies?

\subsection{Solution}

Every drawing of the $a+b$ votes will fall into one of three categories:

\begin{enumerate}
    \item The first tallied vote is for Benjamin; in this case, 
        there is a tie at some point, since Albert ultimately wins
    \item The first tallied vote is for Albert, and he stays ahead 
        during the entire drawing
    \item The first tallied vote is for Albert, but he does not stay 
        strictly ahead during the entire drawing; there is a tie at 
        some point
\end{enumerate}

\vspace{0.5cm}

We will denote votes for Albert and Benjamin as \textit{A} and 
\textit{B}, respectively. Then, an example of a drawing in the 
$1$\textsuperscript{st} category with $a=3$ and $b=2$ is the following:

$$BBAAA$$

In the above example, the $4$\textsuperscript{th} vote tallied 
is the first position  of a tie. As noted earlier, every drawing in the 
$1$\textsuperscript{st} category will contain a tie at some point. 
If we "invert" every vote at or before the position of the first tie 
(\textit{A} becomes \textit{B}, and vice versa), this drawing becomes 
the following:

$$AABB|A$$

We observe now that this "inverted" drawing falls into the 
$3$\textsuperscript{nd} category. This is not an 
anomaly: there is a bijection between the $1$\textsuperscript{st} category
and $3$\textsuperscript{rd} category created by following this inversion 
rule. It follows that the number of drawings in each of these $2$ 
categories is the same, and that their probabilities are also 
the same. This probability is the probability of the first vote 
tallied being for Benjamin, which is $\frac{b}{a+b}$.

Using this observation, we can calculate the probability of 
a random drawing having a tie at some point. Every drawing in the 
$1$\textsuperscript{st} category leads to a tie, since Albert 
must ultimately win ($a > b$). By definition, the
$3$\textsuperscript{rd} category also consists entirely of 
drawings with ties. Thus, the overall probability of a tie is:

$$P(Tie) = \frac{2b}{a+b}$$

This approach of applying inversions to solve the problem is 
called a \textit{Proof by Reflection}.

\end{document}
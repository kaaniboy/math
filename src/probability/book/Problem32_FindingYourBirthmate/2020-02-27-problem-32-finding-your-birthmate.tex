\documentclass{article}
\usepackage{amsmath}
\usepackage{graphicx}
\usepackage[utf8]{inputenc}
\usepackage{parskip}
\usepackage[symbol]{footmisc}

\renewcommand{\thefootnote}{\fnsymbol{footnote}}
\setcounter{secnumdepth}{0}

\newcommand*{\perm}[2]{{}^{#1}\!P_{#2}}%

\date{}
\author{Kaan Aksoy | Feb 27, 2020}

\begin{document}

\maketitle
\section{Finding Your Birthmate}
\subsection{Problem}

You want to find someone whose birthday matches yours. 
What is the least number of strangers whose birthday you 
need to ask about to have a $50$-$50$ chance?

\subsection{Solution}

Let $X$ be a random variable representing the number 
of people you need to ask before finding someone whose 
birthday matches yours. Then, applying the geometric 
distribution, we know $P(X=n)$ is:

$$
P(X=n) = \left(\frac{364}{365}\right)^n
\left(\frac{1}{365}\right)
$$

Extending the above, the probability that $X$ is no more than some value $n$ is simply:

$$
P(X \leq n) = \sum_{i=0}^n \left(\frac{364}{365}\right)^n
\left(\frac{1}{365}\right)
$$


To finish this problem, we assign $P(X \leq n) = \frac{1}{2}$ 
and solve for $n$:

$$
\frac{1}{2} = \sum_{i=0}^n \left(\frac{364}{365}\right)^n
\left(\frac{1}{365}\right) = \frac{\frac{1}{365}\left[1-\left(\frac{364}{365}\right)^n
\right]} {1-\frac{364}{365}}
$$

$$
\frac{1}{2} = \left(\frac{364}{365}\right)^n
$$

The first line above comes from 
applying the formula for the sum of a finite geometric series, 
which is $s_n = \frac{a_1(1-r^n)}{1-r}$. The final answer is 
$\log_\frac{364}{365} \frac{1}{2} \approx 253$.
\end{document}
\documentclass{article}
\usepackage{amsmath}
\usepackage{graphicx}
\usepackage[utf8]{inputenc}
\usepackage{parskip}
\usepackage[symbol]{footmisc}

\renewcommand{\thefootnote}{\fnsymbol{footnote}}
\setcounter{secnumdepth}{0}

\date{}
\author{Kaan Aksoy | March 8, 2020}

\begin{document}

\maketitle
\section{Gambler's Ruin}
\subsection{Problem}

Player $M$ has \$$1$, and Player $N$ has \$$2$. Each play gives 
one of the players \$$1$ from the other. Player $M$ is enough 
better than Player $N$ that he wins $\frac{2}{3}$ of the plays. 
They play until one is bankrupt. What is the chance that 
Player $M$ wins?							

\subsection{Solution}

Let $M$ and $N$ represent victories by Player $M$ and Player $N$, 
respectively. Then, consider the various sequences that 
result in Player $M$'s victory: 

$$\text{Case 0: } M M $$
$$\text{Case 1: } M N M M $$
$$\text{Case 2: } M N M N M M $$
$$ \text{\ldots} $$
$$\text{Case \textit{n}: } (M N)^n M M $$

From the above, we can draw the following 
insights regarding sequences that result in Player $M$'s 
victory:

\vspace{0.3cm}

\begin{itemize}
    \item Player $N$ must not win the $1^{st}$ play
    \item Player $M$ winning $2$ plays in a row results in a victory
    \item Player $N$ must not win $2$ players in a row
\end{itemize}

\vspace{0.3cm}

These insights lead to the following convergent geometric 
series representing Player $M$'s probability of victory: 

$$ P(M\; Victory) = \sum_{i=0}^{\infty} \left[ 
\left( \frac{2}{3} \right) \left( \frac{1}{3}\right) \right] ^n\left(\frac{2}{3}\right)^2 = \frac{4}{7} $$

\end{document}


\documentclass{article}
\usepackage{amsmath}
\usepackage{graphicx}
\usepackage[utf8]{inputenc}
\usepackage{parskip}
\usepackage[symbol]{footmisc}

\renewcommand{\thefootnote}{\fnsymbol{footnote}}

\setcounter{secnumdepth}{0}

\date{}
\author{Kaan Aksoy | March 12, 2020}

\begin{document}

\maketitle
\section{Average Number of Matches}

\subsection{Problem}
The following are two versions of the matching problem:

(a) From a shuffled deck, cards are laid out on a table 
one at a time, face up from left to right, and then another 
deck is laid out so that each of its cards is beneath a 
card of the first deck. What is the average number of 
matches on the card above and the card below in repetitions 
of this experiment?

(b) A typist types letters and envelopes to n different persons. 
The letters are randomly put into the envelopes. On the 
average, how may letters are put into their own envelopes?				

\subsection{Part (a) Solution}

We will apply the property of \textit{Linearity of Expectation}.
Let $X = X_1 + X_2 + \ldots + X_{52}$ be the total number of 
card matches, where each $X_i$ represents the number of 
matches (either $0$ or $1$) for the $i$\textsuperscript{th} 
card. We know $E[X_i] = \frac{1}{52}$. So, we can easily 
calculate the solution as follows:

\begin{equation*}
\begin{split}
E[X] &= E[X_1 + X_2 + \ldots + X_{52}] \\
&= E[X_1] + E[X_2] + \ldots + E[X_{52}] \\
&= 52\left(\frac{1}{52}\right) \\
&= 1
\end{split}
\end{equation*}

In conclusion, we can expect $1$ card to match on average.

\subsection{Part (b) Solution}

The same logic from Part (a) applies here, giving a 
solution of $n\left(\frac{1}{n}\right) = 1$ matching 
letters on average.

\end{document}
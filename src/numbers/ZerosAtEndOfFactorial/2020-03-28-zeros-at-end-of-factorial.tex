\documentclass{article}
\usepackage{amsmath}
\usepackage{graphicx}
\usepackage[utf8]{inputenc}
\usepackage{parskip}
\usepackage[symbol]{footmisc}

\def\lf{\left\lfloor}
\def\rf{\right\rfloor}


\renewcommand{\thefootnote}{\fnsymbol{footnote}}
\setcounter{secnumdepth}{0}

\date{}
\title{}
\author{Kaan Aksoy | Mar 28, 2020}

\begin{document}
\maketitle

\section{Zeros at the End of a Factorial}

\subsection{Problem}

How many trailing zeros does $n!$ have?

\subsection{Solution}

Consider a fairly small factorial, like $10! = 3,628,800$, which 
has $2$ trailing zeros. Is there an easy way to see why it has $2$ 
trailing zeros without performing the multiplications directly? 

$$10! = 10\times9\times8\times7\times6\times5\times4\times3\times2\times1$$

Looking at the above, we can see why $10!$ has $2$ trailing zeros. 
Obviously, the factor $10$ contributes $1$ of these. The other trailing zero 
comes from the product of the factors $5$ and $2$, which also gives $10$.

This idea of counting factors that are multiples of $5$ is a good 
initial approach. Among these, multiples of $10$ immediately contribute 
a trailing zero, while the remaining multiples of $5$ contribute a trailing 
zero when multiplied by an even factor. Since half of the numbers in the 
factorial's expansion are even, there is always an even number to pair 
with a multiple of $5$.

\vspace{0.5cm}

Let's apply this approach to $15!$:
$$
15! = 15\times14\times13\times12\times11\times
10\times9\times8\times7\times6\times5\times4\times3\times2\times1
$$

This factorial contains $3$ multiples of $5$ ($5$, $10$, and $15$), 
which gives a total of $3$ trailing zeros. Performing the 
multiplications gives $15! = 1,307,674,368,000$, which confirms 
the solution.

Now, let's try applying the approach to $25!$. We know the expansion 
of $25!$ contains $5$ multiples of $5$, which are $5$, $10$, $15$, 
$20$, and $25$. Thus, applying our method gives $5$ trailing zeros. 
In reality, however, $25!$ has $6$ trailing zeros:
$$25!= 15,511,210,043,330,985,984,000,000$$

What gives? The issue is that one term in this factorial's expansion 
is $25$ (or $5^2$), which contains two $5$s. Thus, it actually 
contributes $2$ trailing zeros to the count instead of just $1$. This 
shows us that we must account for powers of $5$ specially. For 
example, $5^3=125$ and $5^4=625$ contribute $3$ and $4$ trailing zeros, 
respectively.

Fortunately, we can easily represent this process mathematically. If 
we define $z_n$ as the number of trailing zeros for $n!$, we can write 
the formula as follows:

$$
z_n = \lf\frac{n}{5}\rf + \lf\frac{n}{5^2}\rf + 
\lf\frac{n}{5^3}\rf + \lf\frac{n}{5^4}\rf + \ldots
$$

The above formula only needs to be extended to the largest product 
of $5$ that is less than or equal to $n$. For example, for $125!$ 
we extend it to the $3$\textsuperscript{3rd} term, which gives:

$$
z_{125} = \lf\frac{125}{5}\rf + \lf\frac{125}{5^2}\rf + 
\lf\frac{125}{5^3}\rf = 25 + 5 + 1 = 31
$$



\end{document}
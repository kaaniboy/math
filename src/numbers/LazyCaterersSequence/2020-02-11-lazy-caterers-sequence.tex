\documentclass{article}
\usepackage{amsmath}
\usepackage[utf8]{inputenc}
\usepackage{parskip}%

\setcounter{secnumdepth}{0}

\date{}
\author{Kaan Aksoy | Feb 11, 2020}

\begin{document}

\maketitle

\section{Lazy Caterer's Sequence}
The Lazy Caterer's Sequence describes the maximum number of 
slices that can be made on a pizza using a given number of 
straight cuts. The sequence begins as follows:

\begin{center}
\begin{tabular}{ c|cccccc } 
 Cuts & 0 & 1 & 2 & 3 & 4 & ... \\ 
 \hline
 Slices & 1 & 2 & 4 & 7 & 11 & ... \\ 
\end{tabular}
\end{center}

Intuitively, to maximize slices, each straight cut should 
pass through all prior cuts. Furthermore, that cut should 
not pass through the intersection of two prior cuts. In 
this way, the $n^{th}$ cut will pass through $n-1$ previous 
cuts and be divided into $n$ segments. Each of those 
$n$ segments will split an existing slice in $2$, resulting in 
$n$ new slices.

Formally, this can be expressed by the following recurrence 
relation, with base case $f(0) = 1$. $f(n)$ represents the 
maximum number of slices with $n$ cuts:

$$ f(n) = n + f(n-1) $$

The closed form of this recurrence can be developed using 
the triangle numbers:

\begin{equation*}
\begin{split}
f(n) &= n + f(n-1) \\
&= n + (n-1) + f(n-2) \\
&= n + (n-1) + (n-2) + (n-3) + \ldots + 1 + f(0) \\
&= \frac{n(n+1)}{2} + 1 \\
&= \frac{n^2 + n + 2}{2}
\end{split}
\end{equation*}




\end{document}
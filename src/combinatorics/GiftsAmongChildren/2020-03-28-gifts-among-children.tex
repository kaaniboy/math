\documentclass{article}
\usepackage{amsmath}
\usepackage{graphicx}
\usepackage[utf8]{inputenc}
\usepackage{parskip}
\usepackage[symbol]{footmisc}

\def\lf{\left\lfloor}
\def\rf{\right\rfloor}


\renewcommand{\thefootnote}{\fnsymbol{footnote}}
\setcounter{secnumdepth}{0}

\date{}
\title{}
\author{Kaan Aksoy | Mar 28, 2020}

\begin{document}
\maketitle

\section{Gifts Among Children}

\subsection{Problem}

There are $6$ identical gifts to be distributed among $8$ 
children. Each child may get more than $1$ gift. How many 
ways are there to distribute the $6$ gifts among the children?

\subsection{Solution}

One approach is to apply the \textit{Stars and Bars} method, 
which involves counting the number of ways one can place "bars" 
to divide stars (or in case, gifts) among children.

In this case, there are $8-1=7$ bars, which will divide the stars 
into $8$ buckets (one per child). Additionally, there are 
$6+7=13$ slots where the bars and stars can be placed. Below is 
an example configuration:

$$|\;|\;|\;|\;|**|***|*$$

In the above example, the $1$\textsuperscript{st} -- 5\textsuperscript{th} 
children get no presents, the 6th gets $2$ presents, the 
7\textsuperscript{th} gets $3$ presents, and the 8\textsuperscript{th} 
gets $1$ present.

In general, we can count the number of configurations of $n$ 
indistinguishable objects divided between $k$ buckets as follows:

$$Count = \binom{n+k-1}{k-1}$$

For this particular problem, this method gives the following solution:

$$Count = \binom{6+8-1}{8-1} = \binom{13}{7} = 1716$$
\end{document}
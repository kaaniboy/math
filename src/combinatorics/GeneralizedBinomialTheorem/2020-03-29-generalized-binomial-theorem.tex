\documentclass{article}
\usepackage{amsmath}
\usepackage[utf8]{inputenc}
\usepackage{parskip}%

\setcounter{secnumdepth}{0}

\title{}
\date{}
\author{Kaan Aksoy | Mar 29, 2020}

\begin{document}

\maketitle

\section{Generalized Binomial Theorem}
The \textit{binomial theorem} describes the expansion of powers of 
binomials, and can be stated as follows:

$$(x+y)^n = \sum_{k=0}^n \binom{n}{k}x^k y^{n-k}$$

In the above, $\binom{n}{k}$ represents the number of ways to select 
$k$ objects out of a set of $n$ objects where order does not matter. In 
the context of this theorem, these values are also called 
\textit{binomial coefficients}, and are defined as follows:

$$\binom{n}{k} = \frac{n!}{(n-k)!k!}$$


As example, we can apply this theorem to the expansion of $(x+y)^4$, 
which gives the following:

\begin{equation*}
\begin{split}
(x+y)^4 &= \binom{4}{0}x^0y^4 + \binom{4}{1}x^1y^3 + 
\binom{4}{2}x^2y^2 + \binom{4}{3}x^3y^1 + \binom{4}{4}x^4y^0 \\
&= y^4 + 4xy^3 + 6x^2y^2 + 4x^3y + x^4
\end{split}
\end{equation*}

However, as it stands, this theorem is not directly applicable to 
binomial expansions where the power is negative or not an integer, 
such as $(x+y)^{-1}$ or $(x+y)^{\frac{1}{2}}$. This is because 
the factorial function ($n!$) used in the binomial 
coefficient has a domain restricted to non-negative integers.

Fortunately, it is easy to replace the factorial with an equivalent 
representation to expand the domain:

\begin{equation*}
\begin{split}
\binom{n}{k} &= \frac{n!}{(n-k)!k!} \\
&= \frac{n(n-1)(n-2)\cdots(n-k+1)(n-k)!}{(n-k)!k!} \\ 
&= \frac{n(n-1)(n-2)\cdots(n-k+1)}{k!}
\end{split}
\end{equation*}

We can further simplify this formulation for negative exponents, 
as in $(x+y)^{-n}$, as follows:

\begin{equation*}
\begin{split}
\binom{-n}{k} &= \frac{(-n)(-n-1)(-n-2)\cdots(-n-k+1)}{k!} \\
&= (-1)^k\frac{(n)(n+1)(n+2)\cdots(n+k-1)}{k!} \\ 
&= (-1)^k\frac{(n+k-1)!}{k!(n-1)!} \\
&= (-1)^k\binom{n+k-1}{k}
\end{split}
\end{equation*}

Using these generalized forms, we can describe the expansion of powers 
of binomials whose exponents are any real number, or even complex numbers. 
For example, we can apply them to find the coefficient of $x^{2005}$ in the 
expansion of $( 5x+1)^{-2}$:

\begin{equation*}
\begin{split}
\left[x^{2005}\right](5x+1)^{-2} &= \binom{-2}{2005}5^{2005} \\
&= (-1)^{2005}\binom{2006}{2005}5^{2005} \\
&= -2006 \times 5^{2005}
\end{split}
\end{equation*}

In the above, $\left[\;\cdots\;\right]$ is an operator querying the coefficient 
of its parameter. In this case, the operator is being applied to find the 
coefficient of $x^{2005}$.

Aside from the above use case, the generalized binomial theorem is 
also very useful for \textit{generating functions}, which are heavily 
applied in combinatorics. 

\end{document}